%&tex
\documentclass{article}

\usepackage{listings}
\usepackage{geometry}

\title{Exercise 7 -- DNS lab report}
\author{Stewart Johnston\\
  {CIS 127 -- Intro to Information Security}\\
  {NCMC}\\
  {\texttt{johnstons1@student.ncmich.edu}}
}
\date{\today}

\begin{document}

\maketitle

\section{Domain Name System}
DNS is a complicated procedure, much of which is outside the scope of this
report. The basic principle is as such:

\begin{enumerate}
	\item Numbers are hard for people to remember, by comparison with
		distinct names. DNS servers store the relationship between IP
		addresses and registered domain names.
	\item Every machine needs to retrieve the IP address from a
		DNS server for a domain name when the relationship is not
		already known.
	\item Many home users rely on their gateway (modem/router) to point
		them to a DNS server, which is often provided by the Internet
		Service Provider. Many organizations host their own DNS servers
		which perform the decentralized workings themselves, so as not
		to rely on outside resources which can't be controlled. They
		may also make this server the authoritative DNS server for
		their domain, and make it internet facing.
	\item DNS servers don't retrieve the information fresh from other
		servers every time they need to serve requests; that would be
		monumentally noisy and slow. Instead, they cache results for
		some amount of time.
	\item Malicious attackers can exploit the caching behavior of local
		network DNS servers to point users towards a phony website
		which is designed with hostile intent. One such example is:
		\begin{itemize}
			\item Lookalike sites built to scrape user credentials
				before logging them in. (If I were a threat, I
				would not want my threat discovered quickly. As
				such, I would do everything I could to hide the
				existence of my threat.)
		\end{itemize}
\end{enumerate}

\end{document}

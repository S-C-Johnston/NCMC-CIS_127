%&tex
\documentclass{article}

\usepackage{listings}
\usepackage{geometry}

\title{Exercise 7 -- DNS lab report}
\author{Stewart Johnston\\
  {CIS 127 -- Intro to Information Security}\\
  {NCMC}\\
  {\texttt{johnstons1@student.ncmich.edu}}
}
\date{\today}

\begin{document}

\maketitle

\section{Domain Name System}
	DNS is a complicated procedure, much of which is outside the scope of
	this report. The basic principle is as such:

\begin{enumerate}
	\item Numbers are hard for people to remember, by comparison with
		distinct names. DNS servers store the relationship between IP
		addresses and registered domain names.
	\item Every machine needs to retrieve the IP address from a
		DNS server for a domain name when the relationship is not
		already known.
	\item Many home users rely on their gateway (modem/router) to point
		them to a DNS server, which is often provided by the Internet
		Service Provider. Many organizations host their own DNS servers
		which perform the decentralized workings themselves, so as not
		to rely on outside resources which can't be controlled. They
		may also make this server the authoritative DNS server for
		their domain, and make it internet facing.
	\item DNS servers don't retrieve the information fresh from other
		servers every time they need to serve requests; that would be
		monumentally noisy and slow. Instead, they cache results for
		some amount of time.
	\item Malicious attackers can exploit the DNS protocol to point users
		towards a phony website which is designed with hostile intent.
		One such example is:
		\begin{itemize}
			\item Lookalike sites built to scrape user credentials
				before logging them in. (If I were a threat, I
				would not want my threat discovered quickly. As
				such, I would do everything I could to hide the
				existence of my threat.)
		\end{itemize}
\end{enumerate}

\section{Attacks}
	There are multiple kinds of attacks which can be leveraged on the DNS
	protocol.  Generally, there are two broad methods, with varying degrees
	of longevity and effectiveness.

\subsection{Man in the Middle}
	If the attacker and the victim are on the same LAN, some tools can be
	used to intercept and spoof a legitimate DNS response. The idea is to
	act quickly enough that the first DNS response the victim receives is
	from the attacker.  However, for the attack to work consistently, the
	tool needs to keep listening for requests that the end-user machine
	makes. If the latency between the attacker and the victim is too high,
	and the real DNS responds first, the attack fails. If the attacker
	never hears the request to begin with, such as if the victim is using
	an encrypted VPN tunnel: no dice. This kind of attack could work well
	in situations where there is no locally hosted DNS server, or where
	latency to the server would be wide enough for the attacker to slip in.
	A cafe's wifi network, for instance, would probably be prime territory
	because:
\begin{itemize}
	\item There may be many wireless devices (phones, especially) seeking
		the same websites, and making requests.
	\item It is unlikely that a cafe would be high-tech enough to have a
		locally hosted DNS server. The latency between phone \verb'->'
		router \verb'->' DNS server and back almost certainly leaves a
		wide enough opening for an attack.
\end{itemize}

	The MitM technique serves as an enormous advertisement for VPN services
	and a reminder of the importance of checking certificates for important
	sites.

\subsection{Poisoning the Cache}
	Dumping poison in the watering hole, here meaning the DNS server's
	cache, is very similar to the MitM attack in several ways, but with the
	following distinctions:
	\begin{itemize}
		\item The DNS server must be someplace such the attacker can
			insert their response between the local server and
			other name servers on the internet. It follows that
			this attack may work quite well against an organization
			with the technical chutzpah to host their own DNS
			server, and it would be a rather pointless endeavor in
			nearly any other situation.
		\item The response only need be spoofed {\em once} for any
			given domain, as long as the cache keeps the spoofed
			response. This may be minutes, hours, or days. This has
			an obvious benefit over needing to keep constant watch
			for requests.
		\item Authority and other domains can also be targeted. An
			attacker can point the DNS server to their own
			nameserver elsewhere for the duration that the cache
			keeps that record.
	\end{itemize}

\end{document}

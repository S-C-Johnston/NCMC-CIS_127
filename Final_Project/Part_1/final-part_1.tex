Fullsoft Incorporated recently suffered breach of data confidentiality,
damaging its competitive advantage. Proprietary information was
exfiltrated as a result of malware attack.

This document is an informal collection of thoughts to aid discussion in
an organization meeting, to help make accessible to potential laymen the
security team's expertise. Specifically, the incident itself is to be
analyzed, as well as the impact it may carry.

\hypertarget{circumstances}{%
\section{Circumstances}\label{circumstances}}

Malware can spread through an number of vectors. Malware which is
capable of actually executing and exfiltrating data is likely to take
only a handful of forms.

To lay some groundwork terms: 1. Threat: Any action which can
potentially harm the availability, integrity, or confidentiality of
information. Need not be malicious, merely harmful.

\begin{enumerate}
\def\labelenumi{\arabic{enumi}.}
\setcounter{enumi}{1}
\item
  Internal threat: A threat from inside the organization. This can be as
  simple as a well-meaning user doing something foolish, or as complex
  as a disgruntled employee acting maliciously.

  It is important that we educate users of their responsibility in
  avoiding this, but for morale it is likely best we avoid admonishing
  them to harshly or making them feel stupid. Users are easily made to
  feel stupid just by asking for help from IT or seeing how quickly
  computer professionals solve an issue they were struggling with for
  potentially hours. My experience is that it is best to encourage them
  and provide ample opportunity to overcome that feeling. Users who feel
  inconvenienced or patronized are likely to attempt to circumvent
  security measures, so a balance must be found.
\item
  Remote threat: A threat from outside the organization/network. Remote
  threats are likely to leverage whatever vulnerabilities they find,
  including the goodwill and cooperativeness of individuals in the
  organization.
\end{enumerate}

Because we have limited time and resource, although we would ideally
want to shore up every possible vector, we'll list those most likely to
least likely.

\begin{enumerate}
\def\labelenumi{\arabic{enumi}.}
\item
  Software intentionally but not maliciously installed by a user on the
  network. This is one of the most common threats, is an internal threat
  caused by well-meaning but ignorant users.

  \begin{enumerate}
  \def\labelenumii{\arabic{enumii}.}
  \item
    Far and away the most likely breed of malware is a trojan of some
    kind, or otherwise malicious software smuggled in on the back of
    useful software. They initiate on a network primarily through social
    engineering attacks, such as email-attachments. Trojans are often
    deployed with the intention of using them as a backdoor which tries
    to phone home to the attacker or which the attacker can use to log
    in to the machine remotely.

    \begin{itemize}
    \tightlist
    \item
      Not totally unlikely that a trojan may have been crafted
      specifically for us by a competing organization or criminal.
      Attacking by leaving infected Flash-Drives in a parking lot for
      the curious employee to pick up and plug in is not unheard of.
    \end{itemize}
  \item
    By distinction, a virus is usually less stealthy and usually more
    destructive in behavior, and it typically aims either to disrupt
    operations or make itself known. A virus will usually spread through
    infected files or programs, being attached to some host file, and
    upon running will attempt to infect other host files. Although the
    need to ride on the back of something innocent-looking may seem like
    trojan behavior, the viral, self-replicating nature is what makes it
    a virus, while trojans typically don't attempt this self-replicating
    behavior.
  \end{enumerate}
\item
  A remote attacker may have found a vulnerable internet-facing server
  and through it, pivoted elsewhere. They could leave a rootkit behind,
  allowing them return access. Without interacting with any member of
  staff, they may have been performing reconnaissance for some time. If
  our organization was the subject of corporate espionage, this is not
  an unlikely move.
\end{enumerate}

\hypertarget{similar-incidents}{%
\section{Similar Incidents}\label{similar-incidents}}

Two similar occasions are available to dissect.

\hypertarget{panic}{%
\subsection{Panic}\label{panic}}

The software company \href{https://panic.com}{Panic} suffered theft of
their source code. According to their
\href{https://panic.com/blog/stolen-source-code/}{blog} from one of the
developers, the thieves delivered their attack by disguising their
software as a video-codec processing program called Handbrake.

\begin{enumerate}
\def\labelenumi{\arabic{enumi}.}
\item
  Handbrake is legitimate software, but for a period of three days it
  was vulnerable, and during that time the software was ``nagging them
  for some time to install an update''.
\item
  When the program reported that an incremental update was not
  available, it prompted for a full download of the next version. The
  developer, just wanting to get on with it, didn't stop to think about
  why Handbrake needed elevated privileges to run, or about how sketchy
  the authentication dialogue seemed at the time.
\item
  Git credentials were stolen and used to clone several repositories of
  their source code. Panic was lucky in that the attacker did not get
  clones of every repository, largely because the attacker was guessing
  at their repository names.

  \begin{itemize}
  \tightlist
  \item
    \emph{There is no reason to believe that we are this lucky}.
  \end{itemize}
\item
  They were contacted by the thief to the effect of a ransom. ``Pay us
  or we'll release it into the wild'' is an effective summary. There is
  no reason to believe that once paid the thief will keep their promise.

  \begin{itemize}
  \tightlist
  \item
    These are digital assets which can be copied and redistributed
    easily, not a person or physical goods which must be returned.
  \end{itemize}
\end{enumerate}

They came to three conclusions about the likely impact, which will be
discussed later.

\hypertarget{valve-software}{%
\subsection{Valve Software}\label{valve-software}}

Valve is a games company who suffered
\href{https://games.slashdot.org/story/03/10/02/1547218/half-life-2-source-code-leaked}{theft}
of source code for
\href{https://web.archive.org/web/20040828084259/http://www.halflife2.net:80/forums/showthread.php?s=\&threadid=10692}{Half-Life
2}. At the time of the theft, they had been forced to announce a delay
of the highly anticipated game. The remote threat was not someone
interested in monetary game, but a
\href{https://arstechnica.com/gaming/2016/06/what-drove-one-half-life-2-super-fan-to-hack-into-valves-servers/}{fan}
who had a history of cracking game licenses. It was shared online by
someone with whom the thief had shared his discovery. In any case, the
process involved:

\begin{enumerate}
\def\labelenumi{\arabic{enumi}.}
\item
  The attacker scanned their network, finding vulnerabilities which
  exposed the layout and topology of the internal network.
\item
  Mail accounts were accessed by the attacker, in particular the
  Managing Director's. Valve came to know this by examining their logs.
\item
  Malware had potentially infected the Managing Director's machine.
\item
  Source code was copied and exfiltrated. Apparently only the main trunk
  of the code was grabbed, which was not in itself enough to make a
  playable executable.
\item
  Keyloggers were installed by exploiting a vulnerability in Outlook.
  The malware was a customized version of existing software, this fork
  designed to infect Valve.
\end{enumerate}

\hypertarget{potential-impact}{%
\section{Potential Impact}\label{potential-impact}}

From examining similar accounts, there are a handful of things which may
happen. This list is non-prioritized, since we can't say for sure what
kind of entity was the origin of this threat, or to which extent
information and source code was stolen. If we got lucky, then only a
portion of our code was cloned, but we can't assume this.

\begin{itemize}
\item
  \textbf{Cracked versions of our software are released}, in a word:
  pirating. It is probable this was already happening, to be quite
  honest. If no other damage occurs, this would be a blessing.
\item
  \textbf{Malware infected versions of our apps are released}. As the
  Panic developers noted, if this occurred because of a trojan of
  legitimate software, there is a high risk it opens up our end-users to
  this same kind of attack. We need to take steps to stem the flow and
  invalidate any developer certificates they may have stolen. End-users
  need to be made aware, and if we have any distribution points through
  third-parties, like Apple's App Store, Google Play, or linux community
  package managers, we must double down on working with them to prevent
  unauthorized versions of our software from being released through
  those means.
\item
  \textbf{Knowledge of our program structure allows for further theft},
  especially concerning if there is any user-data which can be accessed
  by someone who merely knows how we retrieve it. This would likely be
  the most damaging to our operations, our end-users, and our
  reputation. If we can stop this from becoming a user-data breach, then
  we should do so as quickly as possible. Possible steps include:

  \begin{itemize}
  \item
    Rotating passwords, certificates, or anything else used to gain
    access to our data.
  \item
    Migrating our data to different servers entirely.
  \item
    Invalidate the old API for end-users to connect. This will put any
    users who do not regularly update their installations of our
    software out of service until they do, but that's a small
    consequence compared to user-data theft.
  \end{itemize}
\item
  \textbf{Competitors use this information to creep up on our
  competitive edge}. As Panic software notes, this clone that the
  attackers have will rapidly go out of date as we continue to develop
  and improve our software. A competitive edge is valuable, but if a
  competitor released software with our fingerprints all over it, that
  would be quite damning. This is also one of the less severe
  consequences we could face, in this professional's opinion. If this is
  all that happens, our best bet is to continue our development and
  release cycle, while dedicating some resources to shoring up our leaks
  and preventing this from happening again.
\end{itemize}

\hypertarget{countermeasures}{%
\section{Countermeasures}\label{countermeasures}}

This list is presented in a very rough order of what is easy, and high
priority. It remains unnumbered because the priority, ease, and
efficiency is not set in stone.

\begin{itemize}
\item
  The first and easiest thing we can put in place immediately is to keep
  our antiviral software up to date and to actually use it frequently.
  Scheduling anti-malware software to run on a daily basis during
  off-hours would be wise. Allowing each person to run anti-malware
  scans on their machine during their lunch break would help catch any
  threats which come in during the morning hours, when many people check
  their email.
\item
  Rotating our current passwords and reviewing our password policy may
  create some productivity hiccups, but it is necessary. If we don't
  have 2 factor authentication in place for members with higher
  privileges, we should.
\item
  Providing security training on a regular basis would be wise. This can
  either be done by delivering a series of videos from an outside
  source, or this can be done in-house.

  \begin{itemize}
  \item
    Outsourcing the bulk of the work, at least initially, can cut down
    on the time our staff needs to spend on the topic, especially given
    that we have other security concerns to which we must attend.
    However, we must be careful that our choice of training does not
    feel patronizing or snide.
  \item
    Doing this in-house would take away labor from other fronts, but
    would allow some benefits. For one, security staff tend to be
    enthusiastic about their topic, and if they can avoid being
    condescending to their peers, they can be very effective in
    transferring that enthusiasm. For another, having a live person
    allows for questions to be asked.
  \item
    Whatever is done, it's important that members are educated on their
    responsibilities, and that we avoid frustrating or patronizing them.
    Generally, the enthusiasm for security will spike for a while after
    training, and will peter out over time, and some people may not be
    engaged by material which feels remedial. For those reasons it is
    important we choose a method which has some novelty and
    repeatability, and we avoid the feeling of managerial meddling.
  \end{itemize}
\item
  A policy which can help prevent suffering from the hijacking of
  legitimate software is to establish an organization software
  distribution point for our internal users. The IT staff can evaluate
  software for any problems and deploy updates when they are determined
  safe to use.

  \begin{itemize}
  \item
    If this proves to be out of our ability, then if software is not
    deploying security updates specifically, it would be wise to wait a
    week to see if any nastiness crops up. Most vulnerabilities are
    discovered within a week of a software update which creates the
    vulnerability, and the bleeding edge is sometimes a dangerous place
    to be. Security updates are the only thing which should be rushed.
  \item
    Additionally, it would be wise to practice policy that software
    installation go through the IT staff. Not all software is something
    we as an organization need or want on our network. This may be
    difficult, but thinking well enough ahead of time to make requests
    and have those requests vetted should be within the capability of a
    software development company.
  \end{itemize}
\item
  Practicing a least-privileges model of security is almost always a
  wise thing to do. The efficacy of this will depend greatly on our
  making daily operations convenient to do without requiring elevated
  privileges. In effect: make sure every user logs into the network and
  into their own machines with only as many privileges as they need.
  BYOD can make this difficult to implement successfully, but can be
  done.
\item
  Speaking of Bring Your Own Device: Full-disk encryption on machines
  which may travel outside our network is a good idea. If someone can
  get physical access to a box, the only way to keep them from owning it
  is to have strong protections on the entirety of its contents, without
  merely relying on the security provided when the OS is loaded.
\item
  Machines storing sensitive information should be kept with both strong
  physical and digital security measures. Full-disk encryption, as
  mentioned, can do wonders, and we use it under the assumption that
  things will go wrong. For physical security, literally bolting things
  down may or may not be overkill. Such measures as chipped security
  cards, or other things in the categories of ``things your have'' or
  ``things you are'' would probably not go amiss. These generally do not
  impact productivity overly much.
\item
  Careful examination of the structure and security of our
  internet-facing machines is warranted, here. If anything is found
  wanting, even if it isn't the attack vector our thief used, it would
  be wise to fill the gaps.
\end{itemize}

\hypertarget{high-level-gap-analysis-plan}{%
\section{High-level gap analysis
plan}\label{high-level-gap-analysis-plan}}

The primary concern of the security team for Fullsoft is the information
security and how to prevent similar breaches from occurring. We know
that proprietary information was copied and leaked as a result of
malware infecting machines. The scope of the analyses laid out here,
therefore, is more concerned with technological aspects resulting in
vulnerabilities to malware and information theft, and digital outages,
rather than physical risks such as power-outages, flooding, etc.

The first part of a gap analysis is understanding the current state of
things.

\begin{enumerate}
\def\labelenumi{\arabic{enumi}.}
\item
  Identify existing conditions. This list remains open to suggestion of
  elements to inspect, within the scope stated above.

  \begin{enumerate}
  \def\labelenumii{\arabic{enumii}.}
  \item
    Who has elevated/admin rights over what parts of the network and
    why?
  \item
    Software installed, why it is installed, and how it is updated.

    \begin{enumerate}
    \def\labelenumiii{\arabic{enumiii}.}
    \tightlist
    \item
      Process for requesting new software, if any. How static are
      software needs to perform operations? Is new 3rd party
      installation a common occurrence?
    \end{enumerate}
  \item
    OSs in use, deployment and update method.

    \begin{enumerate}
    \def\labelenumiii{\arabic{enumiii}.}
    \tightlist
    \item
      Device policy. How are users' own devices handled, if there is a
      Bring Your Own Device policy in place?
    \end{enumerate}
  \item
    Intrusion Detection System/Intrusion Prevention System in use, if
    any.
  \item
    Exfiltration prevention mechanisms, if any.
  \item
    File server/file sharing methods being used. Why and how are they
    being used? What kind of security of is being used?
  \end{enumerate}
\item
  Identify vulnerabilities or other outcomes of those conditions. A risk
  assessment on the current state should be performed at this point.
\item
  Identify the outcome we want as informed by best practices. Keep under
  consideration the balance between availability/convenience and
  security. How much convenience are we willing to sacrifice for the
  sake of security? Does availability/convenience even need to drop at
  all? The following list is subject to any order as determined by
  priorities from the risk assessment.

  \begin{itemize}
  \item
    Who actually needs elevated rights? Principle of least privilege is
    best practice which should be applied liberally.
  \item
    What are the broad strokes of software policy we want for requests,
    updates, etc? What portion should be self-managed vs managed by IT?
  \item
    BYOD policy, and especially with consideration to how much it would
    upset ongoing operations to change it. Declaring that no outside
    devices are to be used for operations may incur some cost if a
    significant number of staff use BYOD.
  \item
    The variety of platforms present in our network, and by extension,
    that IT needs to support. Homogenous networks are easier to manage,
    generally, but are highly subject to the same vulnerabilities across
    the whole network. Heterogenous networks are (in this professional's
    opinion, only marginally) more difficult to manage, but 1
    compromised user or machine does not necessarily mean the whole
    network is compromised.
  \item
    IDS/IPS system of choice.
  \item
    Exfiltration prevention (especially weighed against the difficulty
    to implement). Most research indicates that this is difficult thing
    to do correctly.
  \item
    File sharing needs and how to fulfill them in a secure way.
  \item
    Any other conditions we'd like changed from the present state.
  \end{itemize}
\item
  Once prioritized, going about finding and implementing these solutions
  is the final step.
\end{enumerate}

\hypertarget{risk-assessment-methods}{%
\section{Risk Assessment Methods}\label{risk-assessment-methods}}

\hypertarget{nist-800-30r1}{%
\subsection{NIST 800-30r1}\label{nist-800-30r1}}

The NIST 800-30 (revision 1) is part of a series of publications by the
National Institute of Standards and Technology. It is oriented largely
around threats and vulnerabilities. Its most well-modeled threats are
adversarial threats and weaknesses in security controls.

\hypertarget{octave-allegro}{%
\subsection{OCTAVE Allegro}\label{octave-allegro}}

Carnegie Mellon University's Software Engineering Institute publishes
the OCTAVE risk assessment methods. Allegro is the more distilled
version of this publication. Allegro's scope has a wide-reach, and is
focused on assets first. It covers a broad set of topics with a fairly
general, widely applicable set of tools. It hesitates to go into much
specific detail on any one topic, however.

\hypertarget{recommended-method}{%
\section{Recommended Method}\label{recommended-method}}

I recommend the NIST 800-30 for the risk assessment needs of Fullsoft.

\begin{itemize}
\item
  NIST 800-30 has a heavy focus on adversarial and technological
  threats, which are directly related to information theft and malware.
  The OCTAVE Allegro document also covers this, but it has a much less
  focused goal, and is more broad-reaching, with fewer elements specific
  to adversarial or technological threats.
\item
  The NIST is fairly intuitive. While the appendices and tables used by
  NIST 800-30 are easier to use, the typesetting and structure of the
  OCTAVE Allegro document body is easier to read by comparison. OCTAVE
  Allegro's tables and diagrams have a lot of repeated information
  spread out over more space, and that's where a lot of the actual work
  gets done. OCTAVE Allegro, to the untrained eye, takes a lot of
  cross-referencing and context switching to perform the actual work.
  The NIST 800-30 can be followed fairly linearly once work on the
  assessment starts.
\item
  The OCTAVE Allegro document is 50\% longer than the NIST 800-30. That
  50\% is accounted for mostly in the addition of a series of examples,
  which serve as a good -- but necessary -- demonstration. The OCTAVE
  Allegro process, in my opinion, takes longer to grapple with to learn
  and use. Both documents are quite dry reading, but the NIST has more
  actual meat in it, in my opinion.
\end{itemize}

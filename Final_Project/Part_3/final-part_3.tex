\hypertarget{system-hardening-and-auditing}{%
\section{System hardening and
auditing}\label{system-hardening-and-auditing}}

Microsoft provides a handful of mechanisms to address auditing and
system hardening needs in Windows in their own particular philosophy.
Some of these are of their own doing, and some are inherited from
industry standards set decades ago. Among these are:

\begin{enumerate}
\def\labelenumi{\arabic{enumi}.}
\item
  Access controls and permissions for their filesystem. Sequestering
  users and groups and controlling who can access files can help keep
  both accidental and intentional unauthorized access from occurring.
\item
  Encryption in both a granular sense and for full-disk security. This
  helps to keep sensitive data safe when it is at rest. That is to say,
  when not in use. I know from first hand experience how trivial it is
  to break permissions if I can physically access a machine, since I've
  had to do so to repair machines when the user forgets their
  credentials. Sensitive information should only be stored locally on
  machines if that information is encrypted at rest.
\item
  regedit -- The registry editing tools allow for administrators to
  audit the registry themselves and fix issues they encounter.
\item
  gpedit -- The group policy editing tools help to make managing large
  numbers of computers comparatively easy. This also allows
  administrators to set logging and auditing policies on an entire
  domain at once.
\item
  Event viewer -- What it says on the tin: allows for detailed auditing
  of the binary log data reported from multiple sources within the
  system. Once auditing policy is set, this is where logon events can be
  checked.
\item
  Others outside the scope of this document which help to flesh out the
  toolset.
\end{enumerate}

\hypertarget{antivirus-updates-and-scans}{%
\subsection{Antivirus updates and
scans}\label{antivirus-updates-and-scans}}

AVG is a trusted antivirus company which provides several conveniences
for administrators. Among those is a management console, a tool which
requires that machines be on the same domain as the AVG administration
server. This allows, among other things:

\begin{itemize}
\item
  On-demand and scheduled scans of arbitrarily large groups of machines.
\item
  A centralized destination for warnings about threats found during
  scans, with the option to respond as appropriate.
\item
  A centralized update and upgrade distribution cycle. Instead of every
  machine reaching out to AVG for updates or upgrades, the
  administration server does this as needed, and machines on the domain
  can be served virus definition updates and software upgrades remotely.
\end{itemize}

These things can be done manually on each machine, and for the sake of
argument I'll explain how, but it is tedious. To check for updates in
AVG, one would open the AVG application, locate the help/about section,
and check for virus definition or software updates. It will not
automatically upgrade the software without asking, because this often
requires a computer restart.

Scans can be initiated manually by pressing the big green Scan button,
or they can be scheduled. This can be done in the GUI using Scan Options
-\textgreater{} Manage Scheduled Scans. They can be set for a regular
basis and enabled.

Logs, as far as I can tell, are not exported to be viewed by Windows
Event Viewer unless through the use of AVG CloudCare. Otherwise, their
logs are locally stored in several possible
\href{https://support.avg.com/SupportArticleView?l=en\&urlname=Log-File-Locations-for-AVG-Products}{locations}.
It is not uncommon (though mildly annoying) for antivirus vendors to
require the use of their own tools to view logs.

\hypertarget{logging-and-auditing-logon-events}{%
\subsection{Logging and auditing logon
events}\label{logging-and-auditing-logon-events}}

gpedit, mentioned earlier, comes into play here. This makes it possible
to log on both the domain controller and the workstation both successful
and failed logon events. In gpedit, the details can be found in Computer
Configuration -\textgreater{} Windows Settings -\textgreater{} Security
Settings -\textgreater{} Local Policies -\textgreater{} Audit Policy.

There are two kinds: ``Account Logon'' events and ``Logon'' events.
Microsoft's naming scheme is unfortunate, and it's easy to confuse the
two. There isn't a significant reason not to use both. In short, the
difference is:

\begin{longtable}[]{@{}ll@{}}
\toprule
Account Logon Events & Logon Events\tabularnewline
\midrule
\endhead
Logons for which a machine is the & Logon/off events on any
given\tabularnewline
authority. DCs are the authority for & machine. Logged locally
to\tabularnewline
domain accounts. & that machine.\tabularnewline
\bottomrule
\end{longtable}

How these are viewed, ultimately, is in the event viewer. I have a
tendency to type faster than I think on occasion, so it is not unusual
for me to have two failed logons before a proper success. Using both
Account Logon and Logon events, a success will likely show as two
successes, which is fine. These events can be found under Windows Logs
-\textgreater{} Security.

To filter for only this kind of event, we can handle it in a couple of
ways. Using ``Filter Current Log'' or ``Create Custom View'' on the
right side, we can filter by event ID number code using the
\href{https://docs.microsoft.com/en-us/windows/security/threat-protection/auditing/basic-audit-logon-events}{list}
of
\href{https://docs.microsoft.com/en-us/windows/security/threat-protection/auditing/basic-audit-account-logon-events}{event
codes} that Microsoft publishes. To filter by kind, Logon, rather than
the granular numbered ranges, we need to choose Event Source
-\textgreater{} ``Microsoft Windows security auditing'', and \emph{then}
choose Task Category -\textgreater{} Logon, or any other task categories
desired.

\hypertarget{potentially-problematic-events}{%
\subsection{Potentially problematic
events}\label{potentially-problematic-events}}

I tried this with two machines. On each were different issues. The first
is a workstation, where I noticed that the allowed log file size was
quite small. This is a problem because within 30 minutes, enough noise
was generated in the log files that it pushed other, useful events out
of the log. In this case, Windows was logging noisy machines on the
network which were using multicast DNS every second in an attempt to
find other machines to network with. This was just a handful of
misconfigured machines doing something mostly harmless, but their sheer
volume of chatter pushed out the useful logs I rather wanted to keep.

The other was significantly more worrying. I had a cloud-hosted server
with Windows Server running for several days, which I had forgotten to
shut down. This was hosted through Amazon Web Services, and for servers
of that kind, they recycle IP addresses not in use, so what I was seeing
was not a targeted attack, but likely the result of port sniffing.
Several hundred, if not several thousand, failed logon attempts were
logged, sometimes multiple times a second. The originating IPs were from
a huge range, and each IP would wait a handful of seconds before
retrying. This was likely a botnet which happened to spot my server
online and started hammering away at it. The default login of
Administrator was the target, including other guesses like ``KATIE'',
``NEWUSER'', ``TEACHER'', etc. Thankfully, Amazon's Administrator
passwords on configuration are both quite random and strong.

\hypertarget{what-to-do-about-these-events}{%
\subsection{What to do about these
events}\label{what-to-do-about-these-events}}

The first case is fairly straightforward. Increasing the size allowed
for logging will help prevent spammed errors from disrupting the ability
to audit. The other solutions include reconfiguring the chatty machines.
They were Macintoshes using the Bonjour service, shouting for friends,
which is useful in small domains with no DNS server, but not so useful
in large domains with a properly configured DNS. That's a prime example
of well-meaning default configurations getting in the way. Another
example yet is to use regedit, mentioned earlier, to add an exception
for failures of that kind from those specific IPs. Fixing the
misconfigured machines is the saner solution in the long run, but almost
certainly takes more time than just ignoring those logs or increasing
the log size until that can be achieved.

For the server\ldots{} that's tricky. Because it was a botnet
distributing its attempts over many IPs and many different user names,
that's difficult to guard against without also making it difficult for
myself to log in. The only obvious solutions are also tricky to execute
correctly. Configuring the server and firewall rules to only allow in a
handful of static IPs, such as the static IP that Fullsoft pays for with
business class internet, prevents botnets like this from outside the
network from succeeding. It is wishful thinking that such software as
Fail2Ban would be useful here; since the IPs were so distributed, the
effect would be limited. Disabling the Admin builtin account would be
useful, until such a time as a custom admin account is forgotten or
accidentally locked. The easiest solution which comes to mind
immediately is to set up either private key/passwordless login (fairly
trivial for Linux boxes with SSH), or some variation on using
cryptographic keys resulting in very long, borderline impossible to
guess passwords, and changing those on a regular basis. Besides
automated solutions preventing or avoiding this, configuring the server
to email me when several logon failures are noticed in a short period
would not be unwise.

\hypertarget{hardening-and-auditing-and-towards-fullsofts-goals}{%
\section{Hardening and auditing and towards Fullsoft's
goals}\label{hardening-and-auditing-and-towards-fullsofts-goals}}

Mentioned in previous reports in the fallout of Fullsoft suffering from
malware induced data theft, account and access controls are literally
the least we could do in pursuit of hardening our machines. A principle
of least-privilege applied to user accounts, and logging events
generated is a step up from that. These and a number of other hardening
techniques are relatively trivial to implement, without sacrificing much
in the way of convenience, and providing the IT team more to work with.

We will almost certainly be the target of attacks in the future, whether
these are random from passerby sniffing for vulnerabilities, or
targeted. Logging and auditing processes will allow us to understand
when and how this occurs, and thereby also the frequency.

Some hardening operations require support from the DevOps team, in
particular how proprietary source code is handled. If we use a
distributed version control system such as git, then such techniques as
code signing wouldn't be unwarranted. Forbidding the git server from
serving clone requests from IPs outside our network may be wise. If we
don't already, acquiring business internet service for telecommuting
workers wouldn't be unwise, so as to ensure static IP addresses for them
as well.

\hypertarget{worrying-trends-and-how-they-affect-fullsoft}{%
\subsection{Worrying trends and how they affect
Fullsoft}\label{worrying-trends-and-how-they-affect-fullsoft}}

From both a social engineering and a digital security standpoint, social
media is quite often a dangerous thing. Oversharing information on
social media can disclose secrets, sensitive details about operations,
and details about oneself that are used for security purposes. Security
questions, for instance, should not be used with answer that can be
found on social media. Two suggestions come to mind:

\begin{enumerate}
\def\labelenumi{\arabic{enumi}.}
\item
  Don't post this information, or don't use security questions which are
  based on this information. Use custom security questions, rather than
  your mother's maiden name.
\item
  \emph{LIE!} Nobody but you should know the answer. Telling the truth
  makes it easier to remember, sure, but there's no restriction on you
  lying about your first pet or model of car. As long as you can
  remember it, that's what matters.
\end{enumerate}

\hypertarget{how-this-independent-research-helps-fullsoft}{%
\section{How this independent research helps
Fullsoft}\label{how-this-independent-research-helps-fullsoft}}

First thing that comes to mind in how my projects demonstrated here can
help Fullsoft is in demonstrating how easy it is to configure for
logging, once one knows how. Many if not most or all of the hardening
techniques discussed here can be distributed from the group policy as
set by the domain controller. Techniques like full-disk encryption will
require more labor by the IT team, but are also not out of reach or
impractical. The suggestions explored in this document can and should
all be applied in short order.

\documentclass{article}

\usepackage{geometry}
\usepackage{graphicx}

\title{Exercise 4 -- Nmap challenge}
\author{Stewart Johnston\\
  {CIS 127 -- Intro to Information Security}\\
  {NCMC}\\
  {\texttt{johnstons1@student.ncmich.edu}}
}
\date{}

\begin{document}
\maketitle

\section{Capture 1}
\begin{enumerate}
	\item {\em Most likely scan}: Host discovery scan
	\item {\em Exact syntax of the command}: \verb|nmap -PE -PS443 -PA80 -PP 10.10.10.0/24|
	\item {\em Result of scan}: Shows three other hosts active on that
			subnet, but with filtered ports.
		\item {\em Machine vendors}: Cisco, AMD, Raspberry-Pi
		\item {\em MAC address of the machine running nmap}: AmdPcnet\_af:14:cb
		\item {\em IPv4 address of machine running nmap}: 10.10.10.5
		\item {\em IPv6 address of the machine running nmap}: fe80::c14:ca44:6c33:171
		\item {\em MAC addresses of devices on network}:
		\begin{enumerate}
			\item 10.10.10.5 -- AmdPcnet\_af:14:cb
			\item 10.10.10.1 -- Cisco\_2f:b1:dc
			\item 10.10.10.4 -- AmdPcnet\_94:98:0b
			\item 10.10.10.6 -- Raspberr\_1a:17:61
		\end{enumerate}
\end{enumerate}

\section{Capture 2 -- OS scan}
\begin{enumerate}
	\item {\em Exact syntax of the command}: nmap -sV -O 10.10.10.6
	\item {\em Why are there packets from hosts other than the scanner and
		scanee}? 10.10.10.4 is sending multicast UDP signals, about
		every second, which may be a keepalive signal to a phone.
	\item {\em What services are responding}? 10.10.10.6 responds on SSH, for port 22
	\item {\em What packet numbers are checking telnet}? Packet number 23 sends a SYN on port 23, and packet number 24.
\end{enumerate}

\section{Capture 3}
\begin{enumerate}
	\item {\em Type of scan}: Full-connect TCP scan along top 973 ports with
		some retries and no ping. What's notable about this is that it isn't a
		stealthy scan, because it completes the three-way-handshake
		before resetting, which is an event deemed worthy of logging by
		most systems. It's also the only kind of scan available if you
		don't have elevated permissions, because it uses the OS'
		underlying systemcalls to operate, and therefore behaves the
		legal way that OSs are expected to behave.
	\item {\em Major clue to type of scan}: All rst packets sent by the attacker
		were first preceeded by a full three-way-handshake, of syn,
		syn/ack, ack. There are a number of retransmissions, about half
		a second after the prior attempt to send the packet. I can only
		assume that some attenuation occured on the line, or for some
		other reason the attacker was not receiving the responses
		before sending a retransmission. By a little scripting I've
		determined that the maximum number of retries seen in the
		packet capture file is 2, so I will also assume that is the
		case for the actual command being used. As for no ping, well, there are no ICMP from the attacker.
	\item {\em Exact Syntax}: \verb| nmap -sT -Pn --max-retries 2 --top-ports 973 10.10.10.4 |
	\item {\em What ports are listening on the victim}:
		\begin{enumerate}
			\item Port 139: netbios-ssn
			\item Port 135: epmap
			\item Port 25: SMTP
			\item Port 443: HTTPS
			\item Port 445: microsoft-ds
			\item Port 1119: bnetgame
			\item Port 99: metagram
			\item Port 990: ftps
		\end{enumerate}
\end{enumerate}

\section{Capture 4}
\begin{enumerate}
	\item {\em Most likely type of scan}: No ping, along the top 945 ports, looks like a SYN scan,
		but no rst packets are sent resulting in several
		retransmissions from the victims.
	\item {\em Major clue to type of scan}:
	\item {\em Exact syntax}:
	\item {\em What common service and port number is
		active on 2 of the devices, and which are
		the two}? HTTPS on port 443 is active/open
		at 10.10.10.1 and 10.10.10.4; SSH on port
		22 is active/opent at 10.10.10.1 and
		10.10.10.2
\end{enumerate}

\end{document}

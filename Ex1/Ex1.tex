\documentclass{article}

\usepackage{geometry}

\title{Exercise 1}
\author{Stewart Johnston\\
  {CIS 127 -- Intro to Information Security}\\
  {NCMC}\\
  {\texttt{johnstons1@student.ncmich.edu}}
}
\date

\begin{document}
\maketitle

% Source article: https://www.anitian.com/blog/the-ngfw-is-dead/

% What is a Next-Gen Firewall (NGFW), and what makes it different from a normal firewall?
\section{The Next Generation Firewall}

By definition, a Next Generation Firewall includes the base functionality of a
firewall, so it is more useful to work backwards in definition. A firewall
monitors and controls incoming and outgoing network traffic. First generation
firewalls were more limited in scope, and its security rules were defined in
terms of source, destination, protocol, and port number information as carried
by packets. A NGFW is the result of sticking that and other functionality, like
Intrusion Prevention Systems or proxies in a blender.

A NGFW by comparison uses more advanced filtering techniques. One such
technique is application filtering via deep packet inspection. Deep packet
inspection borders examines in detail the meaty data each packet carries,
rather than just the packet headers. This borders on the invasive, but can be
an effective tool for ensuring the safety of devices on its network. Other
methods include inspecting encrypted traffic, website filtering, Quality of
Service management, and others. Live antivirus or advertisement filtering is
one possible use of technology like this.

% Some information scraped from wikipedia

% What is a network perimeter, and how does the cloud change this?
\subsection{Network perimeter and the cloud}

The network perimter is primarily the notion of a clear traffic and information
boundary between an organization's internal assets and the rest of the world.
As Software as a Service, IoT, and Bring Your Own Device becomes increasingly
ubiquitous, the line of what is an internal asset becomes increasingly blurry.
The access to one's internal assets are increasingly remote and distributed.
More to the point, many workhorse "internal assets" are offered by some
external organization and are not inherently under your own control.

The NGFW and other pieces of hardware like it were what primarily defined the
boundaries of the network perimeter. So many assets are so outside of their
scope of influence that the perimeter has effectively become either too blurry
to consider or broken entirely. Not to mention, other organizations servicing
assets seek to offer NGFW servicing of their own.

% See immun.io/blog/what-is-the-network-perimeter-anyway

% What reasons does the author give to back up his statement that NGFWs don't work?
\subsection{Author's supporting points}

The author of the article supposes that NGFWs are ineffective. They don't fault
the NGFWs themselves, but the structure and scaffolding around them. That
structure results in numerous paths around NGFWs. Third parties wanting
unrestricted access don't want to deal with NGFWs at all. Network
administrators crafting the rules for an NGFW often have to be incredibly
light-handed for fear of blocking legitimate traffic, and angering the users as
a result. Users are already vulnerable, angry users wishing to get around their
restrictions are dangerous. He also speaks to the imbalance of an overwhelming
amount of security alerts and not enough staff to see to them.

% Describe a cloud service that could replace a NGFW.
\subsection{NGFW vs the Cloud}

As mentioned in the article, Amazon, in addition to offering AWS, now also
offers Guard Duty. Guard Duty is an additional service for sale, built to run
and protect the AWS machines and applications a given organization is using.
Buying both the hosting service and the security service from one provider will
allow them to optimize and increase the efficiency of both services. An
in-house NGFW solution, then, will have increasingly diminishing returns.

\end{document}
